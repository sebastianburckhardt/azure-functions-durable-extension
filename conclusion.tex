\section{Conclusion}

We have motivated, explained, implemented, and evaluated a new mechanism called \emph{reactive computations} that makes it easier for developers of actor-based services to quickly and easily propagate changes in actor states to clients that depend on it. Our results show that even though its API is as simple as polling, our fault-tolerant distributed reactive caching algorithm can push changes fully automatically and efficiently. Therefore, it is an attractive alternative to complex and error-prone solutions that implement push-based change propagation explicitly at the application level. 

Much work remains to be done. We believe the performance of our implementation can be further improved, by identifying hot-spots and improving the scheduler. Also, we would like to gather more experience in an industrial context; we have already started a collaboration with a game developer team. Moreover, we would like to explore how to combine reactive computations with other mechanisms, such as streams and event sourcing. Another interesting question is to explore the performance cost of making stronger consistency guarantees, such as causality. Finally, we are working on formalizations for describing the virtual actor model, the reactive caching algorithm, and the consistency guarantees.